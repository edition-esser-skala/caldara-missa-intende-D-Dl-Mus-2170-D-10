\documentclass{ees}

\begin{document}

\eesTitlePage

\eesCriticalReport{
  – & –   & vl     & The directives “T.”, “Vv.” etc. in vl 1/2 indicate
                     the beginning and end of segments where ob 1/2 should
                     play unison with the violins. Based on these directives,
                     the oboe parts of this edition have been assembled.
                     Nevertheless, the directives are retained in vl 1/2. \\
  \midrule
  1 & –    & ob 2  & Bars 45, 146, 152, 169, 178 have been emended to accomodate the oboe’s range.
    & –    & clno  & written sounding in \B1 \\
    & 12   & B     & 12th \sixteenthNote\ in \B1: A16 \\
    & 22   & org   & 3rd \quarterNote\ in \B1: \sharp F8–E8 \\
    & 26   & A     & 4th \quarterNote\ in \B1: \sharp f′4 \\
    & 29   & ob 1, vl 1 & 2nd \quarterNote\ in \B1: 4 × g″16 \\
    & 38   & S     & 2nd \eighthNote\ in \B1: a′8 \\
    & 49   & vla 1 & 2nd \halfNote\ in \B1: \sharp h′2 \\
    & 55–106 & vl  & In \B1, vl 1/2 are indicated by \textit{NB Violini
                     col Organo}. Here, the editor assumed that the violins
                     should play one octave higher than org.
                     Thus, bars 55–59, 61–66, 68, 76, 87, 89, 92–96, 98–103,
                     and 105 were emended to accomodate the violins’ range.
    & 134  & org   & last \eighthNote\ in \B1: a8 \\
    & 138  & clno  & 3rd \quarterNote\ in \B1: \sharp f″8–e″16–d″16 \\
    & 142–178 & –  & According to Zelenka’s directive in \B1,
                     the \textit{Domine Deus, Agnus Dei} should be ommitted
                     (probably because the \textit{Benedictus}
                     is a parody of this movement; see below).
}

\eesToc{}

\eesScore

\end{document}
