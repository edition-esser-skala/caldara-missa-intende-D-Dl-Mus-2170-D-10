\documentclass{ees}

\shorttitle{Missa Intende}

\begin{document}

\eesTitlePage

\eesCriticalReport{
  – & –   & vl     & The directives “T.”, “Vv.” etc. in vl 1/2 indicate
                     the beginning and end of segments where ob 1/2 should
                     play unison with the violins. Based on these directives,
                     the oboe parts of this edition have been assembled.
                     Nevertheless, the directives are retained in vl 1/2. \\
  \midrule
  1 & –    & ob 1  & Bar 312 has been emended to accomodate the oboe’s range. \\
    & –    & ob 2  & Bars 45, 146, 152, 169, 177, and 312 have been emended
                     to accomodate the oboe’s range. \\
    & –    & clno  & written sounding in \B1 \\
    & 12   & B     & 12th \sixteenthNote\ in \B1: A16 \\
    & 22   & org   & 3rd \quarterNote\ in \B1: \sharp F8–E8 \\
    & 26   & A     & 4th \quarterNote\ in \B1: \sharp f′4 \\
    & 29   & ob 1, vl 1 & 2nd \quarterNote\ in \B1: 4 × g″16 \\
    & 38   & S     & 2nd \eighthNote\ in \B1: a′8 \\
    & 49   & vla 1 & 2nd \halfNote\ in \B1: \sharp h′2 \\
    & 55–106 & vl  & In \B1, vl 1/2 are indicated by \textit{NB Violini
                     col Organo}. Here, the editor assumed that the violins
                     should play one octave higher than org.
                     Thus, bars 55–59, 61–66, 68, 76, 87, 89, 92–96, 98–103,
                     and 105 were emended to accomodate the violins’ range. \\
    & 134  & org   & last \eighthNote\ in \B1: a8 \\
    & 138  & clno  & 3rd \quarterNote\ in \B1: \sharp f″8–e″16–d″16 \\
    & 142–178 & –  & According to Zelenka’s directive in \B1,
                     the \textit{Domine Deus, Agnus Dei} should be omitted
                     (probably because the \textit{Benedictus}
                     is a parody of this movement; see below). \\
    & 180  & ob 2, vl 2 & last \quarterNote\ in \B1: e″4 \\
    & 181  & ob 2, vl 2 & 1st \halfNote\ in \B1: a′4 \\
    & 196  & vla 1 & last \quarterNote\ in \B1: a′4 \\
    & 213  & vla 2 & 3rd \halfNote\ in \B1: g2 \\
    & 219  & vla 2 & 1st \halfNote\ in \B1: e2 \\
    & 234  & S     & 2nd \halfNote\ in \B1: c″2 \\
    & 235  & A     & 3rd \halfNote\ in \B1: g′2 \\
    & 247  & vla 1 & 1st \halfNote\ in \B1: \sharp f′2 \\
    & 249  & vla 1 & 1st \halfNote\ in \B1: d′2 \\
    & 264  & vla 2 & 2nd \quarterNote\ in \B1: \sharp c′4 \\
    & 265  & vla 2 & 2nd \halfNote\ in \B1: e′2 \\
    & 265  & T     & 2nd \halfNote\ in \B1: b2 \\
    & 271  & B     & 4th \eighthNote\ in \B1: \sharp f8 \\
    & 279  & vla 1 & 2nd \quarterNote\ in \B1: \sharp g′4 \\
    & 285  & org   & 4th \eighthNote\ in \B1: \sharp C8 \\
    & 288  & S     & last \eighthNote\ in \B1: d″8 \\
    & 291–326 & –  & In \B1, ob 1/2, vl 1/2, and vla 1/2 are indicated by
                     “Li stromenti si canta da le parti”. Accordingly, these
                     instruments were added by the editor. \\
    & 292  & A     & 5th \eighthNote\ in \B1: \sharp f′8 \\
    & 318  & T     & 2nd \quarterNoteDotted\ in \B1: \sharp f′4. \\
  \midrule
  2 & –    & ob 2  & Bars 21, 27, 44, 52, and 67f have been emended
                     to accomodate the oboe’s range. \\*
    & 1–16 & clno  & Since the top margins of \B1 are damaged, only the rhythm
                     of clno may be deduced from the manuscript. The pitches
                     have been largely reconstructed by the editor. \\
    & 17–53 & –    & In \B1, this movement is indicated by “NB Benedictus
                     vide pag: 19 in Gloria”. Since the corresponding movement
                     (i.\,e., \textit{Domine Deus, Agnus Dei}) contains no
                     alternate lyrics, they have been added by the editor. \\
    & 54   & vla 2 & 1st \halfNoteDotted\ in \B1: \quaverRest–b8–g4–b4 \\
    & 55   & clno  & 5th \eighthNote\ in \B1: d″8 \\
  \midrule
  3 & 16   & vl 2  & 3rd/4th \halfNote\ in \B1: b′2–a′2 \\
    & 16   & vla 2 & 3rd \halfNote\ in \B1: g2 \\
    & 27ff & –     & In \B1, the \textit{Dona nobis} is indicated by
                     “Dona nobis vide pag: 14 in Kyrie”. Unfortunately,
                     this \textit{Kyrie} has been lost. \\
}

\eesToc{}

\eesScore

\end{document}
